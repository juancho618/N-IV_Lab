\documentclass{article}
    \usepackage{amssymb}
    \usepackage{color}
    \usepackage{listings}
    \usepackage{graphicx}
    \usepackage{subcaption}
    \usepackage{geometry}
    \usepackage{float}
    \geometry{
    a4paper,
    total={170mm,257mm},
    left=20mm,
    top=20mm,
    }
    
    \setlength{\parindent}{0em}
    \setlength{\parskip}{1em} % length of the spacing
    
    \lstset{ % General setup for the package
        language=Python,
        basicstyle=\small\sffamily,
        numbers=left,
        numberstyle=\tiny,
        frame=tb,
        tabsize=4,
        columns=fixed,
        showstringspaces=false,
        showtabs=false,
        keepspaces,
        commentstyle=\color{red},
        keywordstyle=\color{blue},
        emphstyle=\ttb\color{deepred},    
        stringstyle=\color{deepgreen}
    }
    
    \begin{document}
        \begin{figure}
            \centering
            \includegraphics[width=0.5\linewidth]{../Session 1/img/vub.png}
        \end{figure}
        \title{Navigation and Intelligent Vehicles, Lab session 2 Report}
        \author{Juan Jose Soriano Escobar }
        \maketitle
        \newpage

        \section{Simulated measurement data}
        
        This practical session is a little bit different as the first one in the sense that
        this time we are trying to estimate or to "keep track" of the movement of  an object
        or particle in one given dimension (1-D) with an unpredicted acceleration that could be
        also interpreted as a noise.
        
        In order to simulate the data measurement, it is necessary first to define the movement of
        the particle by a mathematical model as is describe in the equation \ref{eq:1} where the position
        \textbf{\textit{x}} depends on the previous position and velocity plus the noise \textbf{\textit{w}}
        that is given by changes of the acceleration in a given moment for an specific direction.

        \begin{equation}\label{eq:1}
            x(k + 1) = Ax(k) + Gw(k) 
        \end{equation}

        Once the movement of the particle is explained, it is needed to find a way to measure the position of 
        the particle in order to compare and correct the predicted value. The measurement function is given by
        the \ref{eq:2}, which contains \textbf{\textit{x}} as the current movement model of the particle + \textbf{\textit{v}}
        functions that is the noise in the measurement of the position (variable of interest) of the particle in a time 
        \textbf{\textit{k}}.

        \begin{equation}\label{eq:2}
            z(k) = Cx(k) + Hv(k)
        \end{equation}\label{eq:2}

        With the mathematical model defined, it is important to express the generation of the data on a emulated system
        to play with the variables and visualize the results. For this project I used again python and the simulated data is expressed
        in the listing \ref{lst:data}

        \begin{lstlisting}[language=Python, caption= Prediction Kalman function, label={lst:data}]
            data = []
            Xk = np.matrix([[0.0],[10.0]])

            def simulate_movement(A, X, q):            
            process_model = np.dot(A, X) + 
                            np.matrix([[np.random.normal(0, np.sqrt(q*((dt**3)/3)))], 
                            [np.random.normal(0, np.sqrt(q*dt))]]) 
            return process_model
            
            for i in range(100):
                data.append(Xk)
                Xk = simulate_movement(A, Xk, q)
        \end{lstlisting}

    \end{document}